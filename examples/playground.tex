\documentclass{article}
% Symlink tandemtext.sty into this directory for local dev:
%   ln -s ../tandemtext.sty examples/tandemtext.sty
\usepackage[british,spanish,compactlayout]{tandemtext}

\title{tandemtext playground}
\author{Development sandbox}
\date{\today}


\begin{document}
\maketitle

\BilingualTitle{Sample Agreement}{Acuerdo de Ejemplo}

\BilingualSection{Purpose}{Propósito}
\BilingualParagraph{%
This Agreement defines the terms of cooperation. It demonstrates parallel bilingual paragraphs.%
}{%
Este Acuerdo define los términos de cooperación. Demuestra párrafos bilingües paralelos.%
}

\BilingualSubsection{Scope}{Alcance}
\begin{enumerateBilingual}
  \itemBilingual{The Contractor shall deliver materials.}{El Contratista entregará los materiales.}
  \itemBilingual{The Customer shall provide access.}{El Cliente proporcionará el acceso.}
  \itemBilingual{Both parties agree on timelines.}{Ambas partes acuerdan los plazos.}
\end{enumerateBilingual}

\BilingualSubsection{Compact Lists}{Listas compactas}
\begin{itemizeBilingual}
  \itemBilingual{Bullet on the left}{Viñeta a la derecha}
  \itemBilingual{Second bullet}{Segunda viñeta}
\end{itemizeBilingual}

\BilingualSubsection{Custom Spacing}{Espaciado personalizado}
\begin{enumerateBilingual}
  \SetBilingualListSpacing{0pt}{1pt}% parskip, vspace between bilingual items
  \itemBilingual{Tight item A}{Elemento compacto A}
  \itemBilingual{Tight item B}{Elemento compacto B}
\end{enumerateBilingual}

\BilingualSubsection{Tables}{Tablas}
\noindent\begin{tabularx}{\linewidth}{|L{.45\linewidth}|C{.1\linewidth}|R{.45\linewidth}|}
\hline
\BilingualCell{.95\linewidth}{Left label}{Etiqueta derecha} \\ \hline
\end{tabularx}

\BilingualSubsection{Condensed Headings}{Encabezados condensados}
\BilingualSubsubsectionCondensed{Details}{Detalles}
\BilingualParagraph{%
Condensed variants reduce vertical whitespace before and after headings.%
}{%
Las variantes condensadas reducen el espacio vertical antes y después de los encabezados.%
}

\BilingualSection{Dates}{Fechas}

\BilingualParagraph{%
Dates are handled by the \texttt{datetime2} package. All features of \texttt{datetime2} are available; see the package documentation for styles, localisation, and custom formatting.%
}{%
Las fechas se gestionan con el paquete \texttt{datetime2}. Todas las funciones de \texttt{datetime2} están disponibles; consulte la documentación del paquete para estilos, localización y formatos personalizados.%
}

\BilingualParagraph{%
Quick use: ISO-style dates are fully supported and can be displayed by calling the command\\
\texttt{\textbackslash DTMdate\{YYYY-MM-DD\}};\\
this function automatically converts an ISO-formatted date into the appropriate local form. For example,\\
\texttt{\textbackslash DTMdate\{2025-07-10\}}\\
will display the date according to the currently active language settings.%
}{%
Uso rápido: las fechas ISO también son compatibles y pueden mostrarse fácilmente mediante la llamada a la orden\\
\texttt{\textbackslash DTMdate\{YYYY-MM-DD\}};\\
esta función convierte automáticamente una fecha en formato ISO a la forma local correspondiente. Por ejemplo,\\
\texttt{\textbackslash DTMdate\{2025-07-10\}}\\
mostrará la fecha según las convenciones del idioma activo.%
}

\BilingualParagraph{%
Examples: Today is \today. ISO input: \texttt{\textbackslash DTMdate\{2025-07-10\}} → \DTMdate{2025-07-10}. Components: \texttt{\textbackslash DTMdisplaydate\{2025\}\{07\}\{10\}\{-1\}} → \DTMdisplaydate{2025}{07}{10}{-1}.%
}{%
Ejemplos: Hoy es \today. Entrada ISO: \texttt{\textbackslash DTMdate\{2025-07-10\}} → \DTMdate{2025-07-10}. Por componentes: \texttt{\textbackslash DTMdisplaydate\{2025\}\{07\}\{10\}\{-1\}} → \DTMdisplaydate{2025}{07}{10}{-1}.%
}


\end{document}
